\newcommand\nc\newcommand
\nc\rnc\renewcommand

\usepackage{epsfig}
\usepackage{latexsym}
\usepackage{amsmath}
\usepackage{amssymb}

\nc\out[1]{}

\nc\noteOut[2]{\note{#1}\out{#2}}

% While working, use these defs
\nc\note[1]{{\em [#1]}}
\nc\notefoot[1]{\footnote{\note{#1}}}
% But for the submission, use these
%% \nc\note\out
%% \nc\notefoot\out

\nc\todo[1]{\note{To do: #1}}

\nc\figlabel[1]{\label{fig:#1}}
\nc\figref[1]{Figure~\ref{fig:#1}}
\nc\figreftwo[2]{Figures~\ref{fig:#1} and \ref{fig:#2}}

\nc\seclabel[1]{\label{sec:#1}}
\nc\secref[1]{Section~\ref{sec:#1}}
\nc\secreftwo[2]{Sections~\ref{sec:#1} and~\ref{sec:#2}}

\nc\appref[1]{Appendix~\ref{sec:#1}}


%% The name \secdef is already taken
\nc\sectiondef[1]{\section{#1}\seclabel{#1}}
\nc\subsectiondef[1]{\subsection{#1}\seclabel{#1}}
\nc\subsubsectiondef[1]{\subsubsection{#1}\seclabel{#1}}


\nc\needcite{\note{ref}}

% \nc\myurl\texttt


% http://cs.wlu.edu/~necaise/refs/latex2e/env-floats.3.html#lnfigure

% Arguments: env, label, caption, body
\nc\figdefG[4]{\begin{#1}[tbp]
#4
\caption{#3}
\figlabel{#2}
\end{#1}}

% Arguments: label, caption, body
\nc\figdef{\figdefG{figure}}
\nc\figdefwide{\figdefG{figure*}}

%% \nc\figdef[3]{\begin{figure}
%% \centering
%% \frame{#3}
%% \caption{#2}
%% \figlabel{#1}
%% \end{figure}}

%% \nc\figdefwide[3]{\begin{figure*}
%% \centering
%% \frame{#3}
%% \caption{#2}
%% \figlabel{#1}
%% \end{figure*}}

% Arguments: label, caption, body
\nc\figrefdef[3]{\figref{#1}\figdef{#1}{#2}{#3}}

\nc\figrefdefwide[3]{\figref{#1}\figdefwide{#1}{#2}{#3}}


% Image format: PNG or JPEG?  JPEG lets us shrink the files, at some cost
% in fidelity.  Png is much slower to process even when the files are
% smaller.  I guess there's some conversion process going on.
% JPEG compressed at 35x, the figures are smaller and faster to
% process than png.  The eps files are huge (about 80x). 
% Since PNG is lossless, keep the master figures in that format and convert.


\nc\picext{pdf}
%\nc\picext{jpg}
%\nc\picext{eps}
%\nc\picext{tif}

\nc\picfile[1]{figures/#1}
\nc\wpic[2]{\includegraphics[width=#1]{\picfile{#2}}}
\nc\pic[1]{\wpic{3.2in}{#1}}

\nc\wfig[2]{
\begin{center}
\wpic{#1}{#2}
\end{center}
}
\nc\fig[1]{\wfig{4in}{#1}}

\nc\picframe[1]{\pic{#1}}
\nc\picframewide[1]{\wpic{7in}{#1}}

\nc\picdef[2]{\figdef{#1}{#2}{\centering \picframe{#1}}}
\nc\picdefwide[2]{\figdefwide{#1}{#2}{\centering \picframewide{#1}}}

%% \nc\picdef[2]{\begin{figure}
%% \centering
%% \frame{\includegraphics[width=2.7in]{\picfile{#1}}}
%% \caption{#2}
%% \label{fig:#1}
%% \end{figure}}

\nc\picrefdef[2]{\picdef{#1}{#2}\figref{#1}}
\nc\picrefdefwide[2]{\picdefwide{#1}{#2}\figref{#1}}

\nc\picrefdeftwo[4]{\picdef{#1}{#2} \picdef{#3}{#4} \figreftwo{#1}{#3}}

\nc\stats[2]{(\emph{W=#1, D=#2})}

\nc\circuitdef[4]{\picdef{#1}{#2 \stats{#3}{#4}}}
\nc\circuitdefwide[4]{\picdefwide{#1}{#2 \stats{#3}{#4}}}

\nc\circuitrefdef[4]{\circuitdef{#1}{#2}{#3}{#4} \figref{#1}}

\nc\figneeded[1]{\figdef{needed}{#1}}


\newcommand{\stat}[6]{
#1 & #2 & #3 & #4 & #5 & #6 \\ \hline
}
\newcommand{\fftStats}[1]{
\begin{center}
\begin{tabular}{|c|c|c|c|c|c|}
  \hline
  \stat{Type}{$+$}{$\times$}{$-$}{total}{max depth} \hline
  #1
  \hline
\end{tabular}
\end{center}
}

\nc\symTwo[1]{\mathbin{#1\!\!\!#1}}
\nc\symThree[1]{\mathbin{#1\!\!\!#1\!\!\!#1}}


%% \newcommand{\onelinecommentchars}{\quad-{}- }
%% \newcommand{\commentbeginchars}{\enskip\{-}
%% \newcommand{\commentendchars}{-\}\enskip}

\renewcommand{\onelinecommentchars}{-{}- }
\renewcommand{\commentbeginchars}{\{-\,}
\renewcommand{\commentendchars}{-\}\enskip}

\visiblecomments
